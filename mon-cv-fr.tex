% FortySecondsCV LaTeX template
% Copyright © 2019-2020 René Wirnata <rene.wirnata@pandascience.net>
% Licensed under the 3-Clause BSD License. See LICENSE file for details.
%
% Please visit https://github.com/PandaScience/FortySecondsCV for the most
% recent version! For bugs or feature requests, please open a new issue on
% github.
%
% Contributors:
% https://github.com/PandaScience/FortySecondsCV/graphs/contributors
%
% Attributions
% ------------
% * fortysecondscv is based on the twentysecondcv class by Carmine Spagnuolo
%   (cspagnuolo@unisa.it), released under the MIT license and available under
%   https://github.com/spagnuolocarmine/TwentySecondsCurriculumVitae-LaTex
% * further attributions are indicated immediately before corresponding code


%-------------------------------------------------------------------------------
%                             ADDITIONAL PACKAGES
%-------------------------------------------------------------------------------
\documentclass[
	a4paper,
	% 9pt,
	% sidesectionsize=Large,
	% showframes,
	% vline=2.2em,
	% maincolor=cvgreen,
	% sidecolor=gray!50,
	% sidetextcolor=green,
	% sectioncolor=red,
	% subsectioncolor=orange,
	% itemtextcolor=black!80,
	% sidebarwidth=0.4\paperwidth,
	% topbottommargin=0.03\paperheight,
	% leftrightmargin=20pt,
	% profilepicsize=4.5cm,
	% profilepicborderwidth=3.5pt,
	% profilepicstyle=profilecircle,
	% profilepiczoom=1.0,
	% profilepicxshift=0mm,
	% profilepicyshift=0mm,
	% profilepicrounding=1.0cm,
	% logowidth=4.5cm,
	% logospace=5mm,
	% logoposition=before,
	% sidebarplacement=right,
]{FortySecondsCV/fortysecondscv}

% fine tune line spacing
% \usepackage{setspace}
% \setstretch{1.1}

% improve word spacing and hyphenation
\usepackage{microtype}
\usepackage{ragged2e}

% uncomment in case you don't want any hyphenation
% \usepackage[none]{hyphenat}

\usepackage{tabto}

% take care of proper font encoding
\ifxetexorluatex
	\usepackage{fontspec}
	\defaultfontfeatures{Ligatures=TeX}
	% \newfontfamily\headingfont[Path=fonts/]{segoeuib.ttf} % use local font
\else
	\usepackage[utf8]{inputenc}
	\usepackage[T1]{fontenc}
\fi

% use a sans serif font as default
\usepackage[sfdefault]{ClearSans}
% \usepackage[sfdefault]{noto}

% multi-language CV XeLaTeX and polyglossia (should also work with LuaLaTeX)
% NOTE: breaks \pointskill, \membership and some spacings
% \ifxetexorluatex
% 	\usepackage{polyglossia}
% 	\newfontfamily\arabicfontsf[Script=Arabic,Scale=1.5]{Amiri}
% 	\newfontfamily\englishfontsf{Clear Sans}
% 	\setmainfont{Amiri}
% 	\setdefaultlanguage{arabic}
% 	\setotherlanguage{english}
% \fi

% enable mathematical syntax for some symbols like \varnothing
\usepackage{amssymb}

% bubble diagram configuration
\usepackage{smartdiagram}
\smartdiagramset{
	% default font size is \large, so adjust to harmonize with sidebar layout
	bubble center node font = \footnotesize,
	bubble node font = \footnotesize,
	% default: 4cm/2.5cm; make minimum diameter relative to sidebar size
	bubble center node size = 0.4\sidebartextwidth,
	bubble node size = 0.25\sidebartextwidth,
	distance center/other bubbles = 1.5em,
	% set center bubble color
	bubble center node color = maincolor!70,
	% define the list of colors usable in the diagram
	set color list = {maincolor!10, maincolor!40,
	maincolor!20, maincolor!60, maincolor!35},
	% sets the opacity at which the bubbles are shown
	bubble fill opacity = 0.8,
}

%-------------------------------------------------------------------------------
%                            PERSONAL INFORMATION
%-------------------------------------------------------------------------------
%% mandatory information
% your name
\cvname{Christopher\\BRAVO CERCAS}
% job title/career
\cvjobtitle{Informaticien\\[0.2em]}

%% optional information
% profile picture
\cvprofilepic{pics/profile.png}
% logo picture
%\cvlogopic{pics/logo_txt.png}

% NOTE: ordering in sidebar will mimic the following order
% date of birth
\cvbirthday{20 Avril 1985}
% short address/location, use \newline if more than 1 line is required
\cvaddress{13 rue de la Faïencerie F-54400 Longwy}
% phone number
\cvphone{+336 683 114 78}
% personal website
\cvsite{https://cercas.fr}
% email address
\cvmail{christopher@cercas.fr}
% pgp key
%\cvkey{4096R/FF00FF00}{0xAABBCCDDFF00FF00}
% any other custom entry
\cvcustomdata{\faFlag}{Francais}

%-------------------------------------------------------------------------------
%                              SIDEBAR 1st PAGE
%-------------------------------------------------------------------------------
% add more profile sections to sidebar on first page
\addtofrontsidebar{
	% include gosquare national flags from https://github.com/gosquared/flags;
	% naming according to ISO 3166-1 alpha-2 country codes
	\graphicspath{{FortySecondsCV/pics/flags/shiny/}}
	\sidesection{A propos de moi}
		\aboutme{
			Pationné de nouvelles technologie, curieux, j'aime découvrir de nouvelles choses.
		}
	% social network accounts incl. proper hyperlinks
	\sidesection{Réseaux Sociaux}
		\begin{icontable}{2.5em}{1em}
			\social{\faLinkedin}
				{www.linkedin.com/in/cbcercas}
				{cbcercas}
			\social{\faGithub}
				{https://github.com/cbcercas/CV}
				{Github Page CV}
		\end{icontable}

	\sidesection{Linguistiques}
		\pointskill{\flag{FR.png}}{Français}{5}
		\pointskill{\flag{GB.png}}{Anglais}{3}

}


%-------------------------------------------------------------------------------
%                              SIDEBAR 2nd PAGE
%-------------------------------------------------------------------------------
\definecolor{pastelgreen}{HTML}{D7ECD9}
\definecolor{pastelpurple}{HTML}{D5D6EA}
\definecolor{pastelorange}{HTML}{F5D5CB}
\definecolor{pastelyellow}{HTML}{F6F6EB}
\addtobacksidebar{

	\sidesection{Languages}
	\barskill[1ex]{\faSkyatlas}{C}{70}
	\barskill[1ex]{\faSkyatlas}{Python}{60}
	\barskill[1ex]{\faSkyatlas}{AngularJS}{50}

	\sidesection{Os}
	\barskill[1ex]{\faSkyatlas}{Linux}{70}
	\barskill[1ex]{\faSkyatlas}{Freebsd}{60}
	\barskill[1ex]{\faSkyatlas}{Windows}{35}

	\sidesection{Virtualisation}
	\barskill[1ex]{\faSkyatlas}{Esxi}{60}
	\barskill[1ex]{\faSkyatlas}{Proxmox}{30}

	\sidesection{Conteneurisation}
	\barskill[1ex]{\faSkyatlas}{Docker}{70}
	\barskill[1ex]{\faSkyatlas}{Kubernetes}{30}
	\barskill[1ex]{\faSkyatlas}{freebsd Jail}{40}

}


%-------------------------------------------------------------------------------
%                         TABLE ENTRIES RIGHT COLUMN
%-------------------------------------------------------------------------------
\begin{document}

\makefrontsidebar

\cvsection{Formations}
\begin{cvtable}[1.5]
	\cvitem{20015}{Major Digitale}{Paris (75, France)}
		{Formation HEC Major Digitale}
	\cvitem{2009}{Titre professionnel de Maintenance Informatique}{Rennes (35, France)}
		{Équivalent BAC (Reconvertion professionelle)}
	\cvitem{2001 -- 2003}{E.T.E Sanitaire et Thermique}{Nîmes (30, France)}
		{BEP/CAP Plombier chauffagiste}
\end{cvtable}

\cvsection{Expériences Professionels}
\begin{cvtable}[3]
	\cvitem{05-2021 -- ...}{Help Desk}{A.R.H.S Cube S.A  (Belval, Luxembourg)}{}
	\cvitem{Oct 2020 -- Mars 2021}{Developpeur, Devops}{Advanced Biological Laboratories SA  (Luxembourg)}
		{
			\tabto{2mm}- Développement web sous Angular
			\tabto{2mm}- Mise en place  d'une infrastructure réseau compléte (Opnsense, Openvpn, Cisco, Ubiquiti UAP, VLANs...)
			\tabto{2mm}- Migration de firewalls (Watchguard -> Opnsense, IPSec...)
			\tabto{2mm}- Création d'un CI-CD pour des Micro-services (Gitlab, Gitlab-CI, Docker, Nexus, Ansible)
		}
	\cvitem{2020}{Projet personnel Developpeur (En cours)}{}
		{
			Développement d’un wrapper multi-utilisateurs pour poudrière (legestionnaire de compilation des packages FreeBSD).
				\tabto{5mm}- Python
				\tabto{5mm}- VirtualEnv
		}
	\cvitem{2015 - 2018}{Étudiant Ecole 42}{Paris (75, France)}
	{
	42 est une école d'informatique basée sur le peer-learning, offrant différents parcours dans l'apprentissage de la programmation.
		\tabto{2mm}- Branche "Système Unix" ("C"):
   			\tabto{5mm}- libc
			\tabto{5mm}- printf
     		\tabto{5mm}- ls
     		\tabto{5mm}- Shell Posix
	}
	\cvitem{2009 - 2014}{Père au foyer}{Anduze (30, France)}
	{
		Période utilisée pour me former en autodidacte à l'administration système et réseaux ainsi qu'au développement :
		\tabto{2mm}- Création d'une infrastructure sur ESXI, pour mes propres services:
			\tabto{5mm}- Web, mail, DNS...
			\tabto{5mm}- Gestion automatisée grâce à Puppet et Foreman
			\tabto{5mm}- Authentification centralisée avec LDAP et Kerberos
			\tabto{5mm}- Centralisation des données (Logs, trafic...) avec ELK
		\tabto{2mm}- Domotique (Raspberry, Arduino, ESP8266...)
	}
\end{cvtable}





\cvsubsection{Study}
\begin{cvtable}[1.5]
	\cvitem{2006 -- 2008}{Master Studies Panda Science}{Panda Academy}
		{Focus: Advanced rice hat studies and nouveau rain-reflecting cover
		materials.}
	\cvitem{}{Master Theses ($\varnothing\, 1,0$)}{Asian Rice Hat Institute}
		{Impact of solar radiation onto rice hat cover materials with special
		attention to water resistance.}
	\cvitem{2003 -- 2006}{Bachelor Studies PandaScience}{Panda Academy}
		{Focus: Bamboo morphology and its usage in different craftmanships.}
	\cvitem{}{Bachelor Theses ($\varnothing\, 1,0$)}{Bamboo Institute}
		{The bambo flute: An underestimated instrument in orchestras?}
\end{cvtable}

\cvsection{Publications}
\begin{cvtable}
	\cvpubitem{Cooking: 100 recipes for lazy Pandas}{Me and My Panda Friends}
		{Panda's Culinary World}{2010}
	\cvpubitem{Pandastasia}{Still Me}{Bamboo Books Assoc.}{2005}
\end{cvtable}


\cvsection{Awards}
\begin{cvtable}
	\cvitem{2010 -- now}{Panda of the Year}{Panda World Forum}{}
	\cvitem{2005 -- now}{Face of World Wide Fund for Nature}{WWF}{}
	\cvitem{2000}{Winner of Bamboo Sprouts Eating Contest}{Bamboo Society}{}
\end{cvtable}


\cvsection{Extra-Curricular Activities}
\begin{cvtable}
	\cvitemshort{Relaxing}{Master the fine art of relaxing everywhere}
	\cvitemshort{Music}{Playing the bamboo flute in the 1st Panda Orchestra}
	\cvitemshort{Education}{Teaching young pandas to be more panda-like}
\end{cvtable}


\newpage
\makebacksidebar
% \newgeometry{
% 	top=\topbottommargin,
% 	bottom=\topbottommargin,
% 	right=\leftrightmargin,
% 	left=\leftrightmargin
% }

\cvsection{section}
\cvsubsection{Subsection}
\begin{cvtable}
	\cvitem{<dates>}{<cv-item title>}{<location>}{<optional: description>}
\end{cvtable}

\cvsection{cvitem}
\cvsubsection{Multi-line with longer description}
\begin{cvtable}
	\cvitem{date}{Description}{location}{Some longer and more detailed
		description, that takes two lines of space instead of only one.}
	\cvitem{date}{Description}{location}{Some longer and more detailed
		description, that takes two lines of space instead of only one.}
	\cvitem{date}{Description}{location}{Some longer and more detailed
		description, that takes two lines of space instead of only one.}
\end{cvtable}

\cvsubsection{One-line without description}
\begin{cvtable}
	\cvitem{Award}{One-line description}{Sponsor}{}
	\cvitem{Award}{One-line description}{Sponsor}{}
	\cvitem{Award}{One-line description}{Sponsor}{}
\end{cvtable}

\cvsection{cvitemshort}
\cvsubsection{One-line}
\begin{cvtable}
	\cvitemshort{Key}{Some further description}
	\cvitemshort{Key}{Some further description}
	\cvitemshort{Key}{Some further description}
\end{cvtable}

\cvsubsection{Multi-line with longer description}
\begin{cvtable}
	\cvitemshort{Key}{Some further description. Can fill even more than
		only one single line while still keeping the correct indendation level.}
	\cvitemshort{Key}{Some further description. Can fill even more than
		only one single line while still keeping the correct indendation level.}
	\cvitemshort{Key}{Some further description. Can fill even more than
		only one single line while still keeping the correct indendation level.}
\end{cvtable}

\cvsection{cvpubitem}
\begin{cvtable}
	\cvpubitem{Publication title}{Authors}{Journal}{Year}
	\cvpubitem{Publication title}{Authors}{Journal}{Year}
	\cvpubitem{Publication title that is spanning over multiple lines and still
		does not look too bad}{Authors}{Journal}{Year}
\end{cvtable}

\cvsignature

\end{document}
