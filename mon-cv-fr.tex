% FortySecondsCV LaTeX template
% Copyright © 2019-2020 René Wirnata <rene.wirnata@pandascience.net>
% Licensed under the 3-Clause BSD License. See LICENSE file for details.
%
% Please visit https://github.com/PandaScience/FortySecondsCV for the most
% recent version! For bugs or feature requests, please open a new issue on
% github.
%
% Contributors:
% https://github.com/PandaScience/FortySecondsCV/graphs/contributors
%
% Attributions
% ------------
% * fortysecondscv is based on the twentysecondcv class by Carmine Spagnuolo
%   (cspagnuolo@unisa.it), released under the MIT license and available under
%   https://github.com/spagnuolocarmine/TwentySecondsCurriculumVitae-LaTex
% * further attributions are indicated immediately before corresponding code


%-------------------------------------------------------------------------------
%                             ADDITIONAL PACKAGES
%-------------------------------------------------------------------------------
\documentclass[
	a4paper,
	% 9pt,
	% sidesectionsize=Large,
	% showframes,
	% vline=2.2em,
	% maincolor=cvgreen,
	% sidecolor=gray!50,
	% sidetextcolor=green,
	% sectioncolor=red,
	% subsectioncolor=orange,
	% itemtextcolor=black!80,
	% sidebarwidth=0.4\paperwidth,
	% topbottommargin=0.03\paperheight,
	% leftrightmargin=20pt,
	% profilepicsize=4.5cm,
	% profilepicborderwidth=3.5pt,
	% profilepicstyle=profilecircle,
	% profilepiczoom=1.0,
	% profilepicxshift=0mm,
	% profilepicyshift=0mm,
	% profilepicrounding=1.0cm,
	% logowidth=4.5cm,
	% logospace=5mm,
	% logoposition=before,
	% sidebarplacement=right,
]{FortySecondsCV/fortysecondscv}

% fine tune line spacing
% \usepackage{setspace}
% \setstretch{1.1}

% improve word spacing and hyphenation
\usepackage{microtype}
\usepackage{ragged2e}

% uncomment in case you don't want any hyphenation
% \usepackage[none]{hyphenat}

\usepackage{tabto}

% take care of proper font encoding
\ifxetexorluatex
	\usepackage{fontspec}
	\defaultfontfeatures{Ligatures=TeX}
	% \newfontfamily\headingfont[Path=fonts/]{segoeuib.ttf} % use local font
\else
	\usepackage[utf8]{inputenc}
	\usepackage[T1]{fontenc}
\fi

% use a sans serif font as default
\usepackage[sfdefault]{ClearSans}
% \usepackage[sfdefault]{noto}

% multi-language CV XeLaTeX and polyglossia (should also work with LuaLaTeX)
% NOTE: breaks \pointskill, \membership and some spacings
% \ifxetexorluatex
% 	\usepackage{polyglossia}
% 	\newfontfamily\arabicfontsf[Script=Arabic,Scale=1.5]{Amiri}
% 	\newfontfamily\englishfontsf{Clear Sans}
% 	\setmainfont{Amiri}
% 	\setdefaultlanguage{arabic}
% 	\setotherlanguage{english}
% \fi

% enable mathematical syntax for some symbols like \varnothing
\usepackage{amssymb}

% bubble diagram configuration
\usepackage{smartdiagram}
\smartdiagramset{
	% default font size is \large, so adjust to harmonize with sidebar layout
	bubble center node font = \footnotesize,
	bubble node font = \footnotesize,
	% default: 4cm/2.5cm; make minimum diameter relative to sidebar size
	bubble center node size = 0.4\sidebartextwidth,
	bubble node size = 0.25\sidebartextwidth,
	distance center/other bubbles = 1.5em,
	% set center bubble color
	bubble center node color = maincolor!70,
	% define the list of colors usable in the diagram
	set color list = {maincolor!10, maincolor!40,
	maincolor!20, maincolor!60, maincolor!35},
	% sets the opacity at which the bubbles are shown
	bubble fill opacity = 0.8,
}

%-------------------------------------------------------------------------------
%                            PERSONAL INFORMATION
%-------------------------------------------------------------------------------
%% mandatory information
% your name
\ifthenelse{\equal{\detokenize{anonymized}}{\jobname}}{
	\cvname{CBC}
	\cvaddress{****************}
	\cvphone{******************}
	\cvsite{*******************}
	\cvmail{*******************}
}{
	\cvname{Christopher\\BRAVO CERCAS}
	\cvaddress{4 residence Le Moulin F-57270 Richemont}
	\cvphone{+336 683 114 78}
	\cvsite{https://cercas.fr}
	\cvmail{christopher@cercas.fr}
}
% job title/career
\cvjobtitle{Informaticien\\[0.2em]}

%% optional information
% profile picture
\cvprofilepic{pics/profile.png}
% logo picture
%\cvlogopic{pics/logo_txt.png}

% NOTE: ordering in sidebar will mimic the following order
% date of birth
\cvbirthday{20 Avril 1985}

% pgp key
%\cvkey{4096R/FF00FF00}{0xAABBCCDDFF00FF00}
% any other custom entry
\cvcustomdata{\faFlag}{Francais}

%-------------------------------------------------------------------------------
%                              SIDEBAR 1st PAGE
%-------------------------------------------------------------------------------
% add more profile sections to sidebar on first page
\addtofrontsidebar{
	% include gosquare national flags from https://github.com/gosquared/flags;
	% naming according to ISO 3166-1 alpha-2 country codes
	\graphicspath{{FortySecondsCV/pics/flags/shiny/}}
	\sidesection{A propos de moi}
		\aboutme{
			Passionné de nouvelles technologie, curieux, j'aime découvrir de nouvelles choses.
		}
	% social network accounts incl. proper hyperlinks
	\sidesection{Réseaux Sociaux}
		\begin{icontable}{2.5em}{1em}
			\ifthenelse{\equal{\detokenize{anonymized}}{\jobname}}{
				\social{\faLinkedin}
					{********}
					{********}
				\social{\faGithub}
					{********}
					{********}
			}{
				\social{\faLinkedin}
					{www.linkedin.com/in/cbcercas}
					{cbcercas}
				\social{\faGithub}
					{https://github.com/cbcercas/CV}
					{Github Page CV}
			}
		\end{icontable}

	\sidesection{Linguistiques}
		\pointskill{\flag{FR.png}}{Français}{5}
		\pointskill{\flag{GB.png}}{Anglais}{3}

}


%-------------------------------------------------------------------------------
%                              SIDEBAR 2nd PAGE
%-------------------------------------------------------------------------------
\definecolor{pastelgreen}{HTML}{D7ECD9}
\definecolor{pastelpurple}{HTML}{D5D6EA}
\definecolor{pastelorange}{HTML}{F5D5CB}
\definecolor{pastelyellow}{HTML}{F6F6EB}
\addtobacksidebar{

	\sidesection{Languages}
	\barskill[1ex]{\faSkyatlas}{C}{70}
	\barskill[1ex]{\faSkyatlas}{Python}{60}
	\barskill[1ex]{\faSkyatlas}{AngularJS}{50}

	\sidesection{Os}
	\barskill[1ex]{\faSkyatlas}{Linux}{70}
	\barskill[1ex]{\faSkyatlas}{Freebsd}{60}
	\barskill[1ex]{\faSkyatlas}{Windows}{35}

	\sidesection{Virtualisation}
	\barskill[1ex]{\faSkyatlas}{Esxi}{60}
	\barskill[1ex]{\faSkyatlas}{Proxmox}{30}

	\sidesection{Conteneurisation}
	\barskill[1ex]{\faSkyatlas}{Docker}{70}
	\barskill[1ex]{\faSkyatlas}{Kubernetes}{30}
	\barskill[1ex]{\faSkyatlas}{freebsd Jail}{40}

}


%-------------------------------------------------------------------------------
%                         TABLE ENTRIES RIGHT COLUMN
%-------------------------------------------------------------------------------
\begin{document}

\makefrontsidebar

\cvsection{Formations}
\begin{cvtable}[1.5]
	\cvitem{2015}{Major Digitale}{Paris (75), France}
		{Formation HEC Major Digitale}
	\cvitem{2009}{Titre professionnel de Maintenance Informatique}{Rennes (35), France}
		{Équivalent BAC (Reconvertion professionelle)}
	\cvitem{2001 -- 2003}{E.T.E Sanitaire et Thermique}{Nîmes (30), France}
		{BEP/CAP Plombier chauffagiste}
\end{cvtable}

\cvsection{Expériences Professionels}
\begin{cvtable}[3]
	\cvitemlong{10-2021 - 10-2023}{Ingénieur Systéme Linux junior}{Capelen, Luxembourg}{Luxtrust S.A}
 	{
  		\tabto{2mm}- Maintenance de serveurs Linux RedHat
    		\tabto{2mm}- Conteneurisation docker, docker-compose, podman
      		\tabto{2mm}- Automatisation Jenkins, ansible
		\tabto{2mm}- Script bash, python
	}
	\cvitemlong{05-2021 - 09-2021}{Help Desk}{Belval, Luxembourg}{A.R.H.S Cube S.A}
 	{
  		\tabto{2mm}- Helpdesk
	}
	\cvitemlong{Oct 2020 - Mars 2021}{Développeur, Devops}{Luxembourg}{Advanced Biological Laboratories SA}
		{
			\tabto{2mm}- Développement web sous Angular
			\tabto{2mm}- Mise en place  d'une infrastructure réseau compléte (Opnsense, Openvpn, Cisco, Ubiquiti UAP, VLANs...)
			\tabto{2mm}- Migration de firewalls (Watchguard -> Opnsense, IPSec...)
			\tabto{2mm}- Création d'un CI-CD pour des Micro-services (Gitlab, Gitlab-CI, Docker, Nexus, Ansible)
		}
	\cvitemlong{2015 - 2018}{Étudiant}{Paris (75), France}{Ecole 42}
	{
	42 est une école d'informatique basée sur le peer-learning, offrant différents parcours dans l'apprentissage de la programmation.
		\tabto{2mm}- Branche "Système Unix" ("C"):
   			\tabto{4mm}- libc
			\tabto{4mm}- printf
     		\tabto{4mm}- ls
     		\tabto{4mm}- Shell Posix
	}
	\cvitem{2009 - 2014}{Père au foyer}{Anduze (30), France}
	{
		Période utilisée pour me former en autodidacte à l'administration système et réseaux ainsi qu'au développement :
		\tabto{2mm}- Création d'une infrastructure sur ESXI, pour mes propres services:
			\tabto{4mm}- Web, mail, DNS...
			\tabto{4mm}- Gestion automatisée grâce à Puppet et Foreman
			\tabto{4mm}- Authentification centralisée avec LDAP et Kerberos
			\tabto{4mm}- Centralisation des données (Logs, trafic...) avec ELK
		\tabto{2mm}- Domotique (Raspberry, Arduino, ESP8266...)
	}
	\cvitemlong{Juin 2009 - Dec 2009}{Stagiaire Administrateur système}{St Nicolas-de-Redon (44), France}{Alexander Trade Computer}
	{
		Création d'une infrastructure système :
			\tabto{2mm}- Ipcop
			\tabto{2mm}- Windows Serveur 2003 (logiciel de gestion client, fournisseurs, stock, sauvegardes...)
			\tabto{2mm}- Authentification centralisée avec Active Directory
			\tabto{2mm}- VPN employés et clients
			\tabto{2mm}- PXE (installation, sauvegarde, utilitaires de dépannage...)
	}
	\cvitem{2004 - 2008}{Divers contrats}{(35, 44), France}
	{
		Technicien de maintenance informatique particuliers et professionnels, Manutentionnaire, Démolition, Agent de production agro-alimentaire, Plombier chauffagiste, Electricien.
	}
	\cvitem{2003}{Militaire du Rang}{Hyères-les-Palmiers (83), France}
	{ }
\end{cvtable}

\newpage
\makebacksidebar
% \newgeometry{
% 	top=\topbottommargin,
% 	bottom=\topbottommargin,
% 	right=\leftrightmargin,
% 	left=\leftrightmargin
% }

\cvsection{Expériences Personnel}
\begin{cvtable}
	\cvitem{2020}{Developpeur (En cours)}{}
	{
		Développement d’un wrapper multi-utilisateurs pour poudrière (legestionnaire de compilation des packages FreeBSD).
			\tabto{2mm}- Python
			\tabto{2mm}- VirtualEnv
			\\
	}
	\cvitem{2014 - 2019}{Formation informatique pour aveugle}{Anduze (30), France}
	{
		\tabto{2mm}- NVDA: lecteurs d'écran opensource
		\tabto{2mm}- Gestion fichiers
		\tabto{2mm}- Navigation internet, recherche de livre vocaux, téléchargement...
		\tabto{2mm}- Audacity: Numerisation de vinyles et édition
		\tabto{2mm}- Terminal linux plus adapté pour la gestion de fichers 
	}
\end{cvtable}

\cvsignature

\end{document}
